\documentclass[12pt]{article}
\usepackage{lingmacros}
\usepackage{tree-dvips}
\begin{document}
\pagenumbering{gobble}
\title{\bf{White Dwarf Mergers on Adaptive Meshes}}

\vspace*{3\baselineskip}
\centerline{\bf{White Dwarf Mergers on Adaptive Meshes}}
\vspace*{1\baselineskip}
\centerline{A Dissertation presented}
\vspace*{1\baselineskip}
\centerline{by} 
\vspace*{1\baselineskip}
\centerline{\bf{Maximilian Peter Katz}}
\vspace*{1\baselineskip}
\centerline{to} 
\vspace*{1\baselineskip}
\centerline{The Graduate School}
\vspace*{1\baselineskip}
\centerline{in Partial Fulfillment of the}
\vspace*{1\baselineskip}
\centerline{Requirements}
\vspace*{1\baselineskip}
\centerline{for the Degree of}
\vspace*{1\baselineskip}
\centerline{\bf{Doctor of Philosophy}}
\vspace*{1\baselineskip}
\centerline{in}
\vspace*{1\baselineskip}
\centerline{\bf{Physics}}
\vspace*{2\baselineskip}
\centerline{Stony Brook University}
\vspace*{2\baselineskip}
\centerline{\bf{August 2016}}     

\newpage
\pagenumbering{roman}
\setcounter{page}{2}

\centerline{\bf{Stony Brook University}}
\vspace*{1\baselineskip}
\centerline{The Graduate School}
\vspace*{2\baselineskip}
\centerline{Maximilian Peter Katz}
\vspace*{2\baselineskip}
\centerline{We, the dissertation committe for the above candidate for the}
\vspace*{1\baselineskip}
\centerline{Doctor of Philosophy degree, hereby recommend}
\vspace*{1\baselineskip}
\centerline{acceptance of this dissertation}
\vspace*{2\baselineskip}
\centerline{\bf{Michael Zingale - Dissertation Advisor}}
\centerline{\bf{Associate Professor, Physics and Astronomy}}
\vspace*{2\baselineskip}
\centerline{\bf{Alan Calder - Chairperson of Defense}}
\centerline{\bf{Associate Professor, Physics and Astronomy}}
\vspace*{2\baselineskip}
\centerline{\bf{Joanna Kiryluk - Committee Member}}
\centerline{\bf{Assistant Professor, Physics and Astronomy}} 
\vspace*{2\baselineskip}
\centerline{\bf{Marat Khairoutdinov - External Committee Member}}
\centerline{\bf{Associate Professor, School of Marine and Atmospheric Sciences}}
\vspace*{2\baselineskip}
\centerline{This dissertation is accepted by the Graduate School}
\vspace*{3\baselineskip}
\centerline{Charles Taber}
\centerline{Dean of the Graduate School}

\newpage

\centerline{Abstract of the Dissertation}
\vspace*{1\baselineskip}
\centerline{\bf{White Dwarf Mergers on Adaptive Meshes}}
\vspace*{1\baselineskip}
\centerline{by}
\vspace*{1\baselineskip}
\centerline{\bf{Maximilian Peter Katz}}
\vspace*{1\baselineskip}
\centerline{\bf{Doctor of Philosophy}}
\vspace*{1\baselineskip}
\centerline{in}
\vspace*{1\baselineskip}
\centerline{\bf{Physics}}
\vspace*{1\baselineskip}
\centerline{Stony Brook University}
\vspace*{1\baselineskip}
\centerline{\bf{2016}}
\vspace*{2\baselineskip}
The mergers of binary white dwarf systems are potential progenitors of astrophysical
explosions such as Type Ia supernovae. These white dwarfs can merge either by orbital
decay through the emission of gravitational waves or by direct collisions as a result of
orbital perturbations. The coalescence of the stars may ignite nuclear fusion, resulting in
the destruction of both stars through a thermonuclear runaway and ensuing detonation.
The goal of this dissertation is to simulate binary white dwarf systems using the
techniques of computational fluid dynamics and therefore to understand what numerical
techniques are necessary to obtain accurate dynamical evolution of the system, as well as
to learn what conditions are necessary to enable a realistic detonation. For this purpose I
have used software that solves the relevant fluid equations, the Poisson equation for self-
gravity, and the systems governing nuclear reactions between atomic species. These
equations are modeled on a computational domain that uses the technique of adaptive
mesh refinement to have the highest spatial resolution in the areas of the domain that are
most sensitive to the need for accurate numerical evolution. I have identified that the
most important obstacles to accurate evolution are the numerical violation of
conservation of energy and angular momentum in the system, and the development of
numerically seeded thermonuclear detonations that do not bear resemblance to physically
correct detonations. I then developed methods for ameliorating these problems, and
determined what metrics can be used for judging whether a given white dwarf merger
simulation is trustworthy. This involved the development of a number of algorithmic
improvements to the simulation software, which I describe. Finally, I performed high-
resolution simulations of typical cases of white dwarf mergers and head-on collisions to
demonstrate the impacts of these choices. The results of these simulations and the
corresponding implications for white dwarf mergers as astrophysical explosion
progenitors are discussed.

\newpage
\centerline{\bf{Dedication Page}}
\vspace*{4\baselineskip}
This page is optional.

\newpage
\centerline{\bf{Frontispiece}}
\vspace*{4\baselineskip}
The frontispiece is generally an illustration, and is an optional page.

\newpage
\centerline{\bf{Table of Contents}}

\newpage
\centerline{\bf{List of Figures/Tables/Illustrations}}
\vspace*{4\baselineskip}
Include these lists if applicable - if you have a list for each, each list must begin on a new page.

\newpage
\centerline{\bf{List of Abbreviations}}
\vspace*{4\baselineskip}
Include this list if applicable.

\newpage
\centerline{\bf{Acknowledgements}}
\vspace*{4\baselineskip}
When I graduated from high school, the event that mattered most to me was 
not walking across the stage at commencement, but rather the end-of-year 
debate team dinner and awards ceremony. The group of people that had been
my friends on the debate team made my experience much more memorable and 
gave me a connection to the school that I would not otherwise have had.
While I have lost touch with many of the people I knew at the time, I am 
still very close friends with my friends from debate. Andrew, Chris D.,
Chris V., Chris K., Lucas, and the rest of our group of friends have 
what we have in large part because of that activity, and that is what 
really made my educational activities complete.

In the same way, while I have been honored to have the opportunity to 
earn a doctorate from Stony Brook University, what really has made my 
time here special is the group of people I regularly interacted with 
on the astronomy floor. My advisor, Mike Zingale, is of course brilliant 
and someone whose talent level is extraordinary enough that I hope to 
one day reach even a fraction of that level. But I did not come to Stony 
Brook for that reason; there are smart and talented people everywhere. 
I came here and put in the effort to complete this dissertation because 
Mike inspires me to do better and constantly challenge myself;
and, because he shares many of my values about what makes good science and 
how good software development for scientific computing should be done;
but most importantly, because he has the necessary sense of humor to work 
with me. The other half of the dynamic duo that brought me here is Alan 
Calder, and he too played a key role in my success. Alan never let me 
doubt that I was capable of producing great scientific work, and would 
always drop whatever he was doing to chat with me about my concerns about 
my work or about my future. Many people have academic advisors, but some 
of them go above and beyond the call of duty to be true mentors and friends,
and Alan does that every day, not just for me but for all of the students 
he works with. Doug Swesty also played a key role in the development of the 
project that I worked on and in giving me many pointers on how to proceed 
in a novel way when I felt stuck. Even when he couldn't solve my problem, 
he always could provide some interesting insight into it, because there 
are few sharper and more thoroughly knowledgeable people out there.

Then were the students on the floor who made every day a delight, and who
were always willing to chat when I walked over to bother them (in some cases
a little \textit{too} willing, as anyone who was present for one of my many
silly arguments with Rahul can attest). Rahul and Melissa L. are good friends 
who helped keep the astronomy group anchored, and I am glad I got the 
opportunity to enjoy their presence. Adam and Don have a wealth of knowledge 
about subjects that nicely complements my own and working with them helped my 
own productivity enormously. Melissa H. also helped my productivity by doing 
a project that I never had time for -- and she is also a dear friend. Mat 
was also a great office mate, the few times we were actually in the office 
together. Thanks for sharing death row with me, buddy. And to everyone else
-- Lupe, Tianqi, Taeho, and everyone else who spent any time in the grad 
suite -- thanks for always being there to entertain my thoughts.

In my speech at the aforementioned debate dinner, I mentioned that my 
parents were the largest factor in my success, and that's just as true 
now as it was back then. I am truly fortunate to have parents who without 
hesitation will always talk to me, or let me crash at their house on 
last minute notice, or help me out financially when I need it. I don't 
see them or talk to them nearly as much as they would probably like, 
but they are always in my thoughts and I know that I am always in theirs.
And aside from the people already mentioned, Andrea played the most significant 
role in shaping the adult that I have become, and I am grateful to have known 
her for many years and to call her my friend.

Finally, I would like to acknowledge the support of Ann Almgren and Weiqun 
Zhang at Lawrence Berkeley National Laboratory's Center for Computational 
Science and Engineering. They were core developers of the software that 
I used for the simulations in this work, but they were also wonderful resources 
for sounding off new ideas and figuring out how to implement many of the 
algorithms I considered over the years.

\newpage
\pagenumbering{arabic}
\centerline{\bf{Chapter 1}}

\end{document}
