\documentclass[12pt]{article}
\usepackage[margin=1.0in]{geometry}
\usepackage{lingmacros}
\usepackage{tree-dvips}
\usepackage[nottoc]{tocbibind}
\usepackage{natbib}
\usepackage{hyperref}
\begin{document}
\pagenumbering{gobble}

\newcommand\aj{Astronomical Journal}%        % Astronomical Journal 
\newcommand\araa{Annual Review of Astronomy and Astrophysics}%  % Annual Review of Astron and Astrophys 
\newcommand\apj{Astrophysical Journal}%    % Astrophysical Journal 
\newcommand\apjl{Astrophysical Journal Letters}%     % Astrophysical Journal, Letters % MK Edit 2016-01-15: the macro above didn't seem to be defined
\newcommand\apjs{Astrophysical Journal Supplement}%    % Astrophysical Journal, Supplement 
\newcommand\apss{Astrophysics and Space Science}%  % Astrophysics and Space Science 
\newcommand\aap{Astronomy and Astrophysics}%     % Astronomy and Astrophysics 
\newcommand\aapr{Astronomy and Astrophysics Reviews}%  % Astronomy and Astrophysics Reviews 
\newcommand\aaps{Astronomy and Astrophysics Supplement}%    % Astronomy and Astrophysics, Supplement 
\newcommand\icarus{Icarus}% % Icarus
\newcommand\mnras{Monthly Notices of the Royal Astronomical Society}%   % Monthly Notices of the RAS 
\newcommand\pra{Phys.~Rev.~A}% % Physical Review A: General Physics 
\newcommand\prb{Phys.~Rev.~B}% % Physical Review B: Solid State 
\newcommand\prc{Phys.~Rev.~C}% % Physical Review C 
\newcommand\prd{Phys.~Rev.~D}% % Physical Review D 
\newcommand\pre{Phys.~Rev.~E}% % Physical Review E 
\newcommand\prl{Phys.~Rev.~Lett.}% % Physical Review Letters 
\newcommand\actaa{Acta Astronomica}%  % Acta Astronomica

% Override choices in \autoref
\def\sectionautorefname{Section}
\def\subsectionautorefname{Section}
\def\subsubsectionautorefname{Section}

\newcommand{\msolar}{\mathrm{M}_\odot}

% Software names
\newcommand{\boxlib}{\texttt{BoxLib}}
\newcommand{\castro}{\texttt{CASTRO}}
\newcommand{\wdmerger}{\texttt{wdmerger}}
\newcommand{\python}{\texttt{Python}}
\newcommand{\matplotlib}{\texttt{matplotlib}}
\newcommand{\yt}{\texttt{yt}}

\title{\bf{White Dwarf Mergers on Adaptive Meshes}}

\vspace*{3\baselineskip}
\centerline{\bf{White Dwarf Mergers on Adaptive Meshes}}
\vspace*{1\baselineskip}
\centerline{A Dissertation presented}
\vspace*{1\baselineskip}
\centerline{by} 
\vspace*{1\baselineskip}
\centerline{\bf{Maximilian Peter Katz}}
\vspace*{1\baselineskip}
\centerline{to} 
\vspace*{1\baselineskip}
\centerline{The Graduate School}
\vspace*{1\baselineskip}
\centerline{in Partial Fulfillment of the}
\vspace*{1\baselineskip}
\centerline{Requirements}
\vspace*{1\baselineskip}
\centerline{for the Degree of}
\vspace*{1\baselineskip}
\centerline{\bf{Doctor of Philosophy}}
\vspace*{1\baselineskip}
\centerline{in}
\vspace*{1\baselineskip}
\centerline{\bf{Physics}}
\vspace*{2\baselineskip}
\centerline{Stony Brook University}
\vspace*{2\baselineskip}
\centerline{\bf{August 2016}}     

\newpage
\pagenumbering{roman}
\setcounter{page}{2}

\centerline{\bf{Stony Brook University}}
\vspace*{1\baselineskip}
\centerline{The Graduate School}
\vspace*{2\baselineskip}
\centerline{Maximilian Peter Katz}
\vspace*{2\baselineskip}
\centerline{We, the dissertation committe for the above candidate for the}
\vspace*{1\baselineskip}
\centerline{Doctor of Philosophy degree, hereby recommend}
\vspace*{1\baselineskip}
\centerline{acceptance of this dissertation}
\vspace*{2\baselineskip}
\centerline{\bf{Michael Zingale - Dissertation Advisor}}
\centerline{\bf{Associate Professor, Physics and Astronomy}}
\vspace*{2\baselineskip}
\centerline{\bf{Alan Calder - Chairperson of Defense}}
\centerline{\bf{Associate Professor, Physics and Astronomy}}
\vspace*{2\baselineskip}
\centerline{\bf{Joanna Kiryluk - Committee Member}}
\centerline{\bf{Assistant Professor, Physics and Astronomy}} 
\vspace*{2\baselineskip}
\centerline{\bf{Marat Khairoutdinov - External Committee Member}}
\centerline{\bf{Associate Professor, School of Marine and Atmospheric Sciences}}
\vspace*{2\baselineskip}
\centerline{This dissertation is accepted by the Graduate School}
\vspace*{3\baselineskip}
\centerline{Charles Taber}
\centerline{Dean of the Graduate School}

\newpage

\centerline{Abstract of the Dissertation}
\vspace*{1\baselineskip}
\centerline{\bf{White Dwarf Mergers on Adaptive Meshes}}
\vspace*{1\baselineskip}
\centerline{by}
\vspace*{1\baselineskip}
\centerline{\bf{Maximilian Peter Katz}}
\vspace*{1\baselineskip}
\centerline{\bf{Doctor of Philosophy}}
\vspace*{1\baselineskip}
\centerline{in}
\vspace*{1\baselineskip}
\centerline{\bf{Physics}}
\vspace*{1\baselineskip}
\centerline{Stony Brook University}
\vspace*{1\baselineskip}
\centerline{\bf{2016}}
\vspace*{2\baselineskip}
The mergers of binary white dwarf systems are potential progenitors of astrophysical
explosions such as Type Ia supernovae. These white dwarfs can merge either by orbital
decay through the emission of gravitational waves or by direct collisions as a result of
orbital perturbations. The coalescence of the stars may ignite nuclear fusion, resulting in
the destruction of both stars through a thermonuclear runaway and ensuing detonation.
The goal of this dissertation is to simulate binary white dwarf systems using the
techniques of computational fluid dynamics and therefore to understand what numerical
techniques are necessary to obtain accurate dynamical evolution of the system, as well as
to learn what conditions are necessary to enable a realistic detonation. For this purpose I
have used software that solves the relevant fluid equations, the Poisson equation for self-
gravity, and the systems governing nuclear reactions between atomic species. These
equations are modeled on a computational domain that uses the technique of adaptive
mesh refinement to have the highest spatial resolution in the areas of the domain that are
most sensitive to the need for accurate numerical evolution. I have identified that the
most important obstacles to accurate evolution are the numerical violation of
conservation of energy and angular momentum in the system, and the development of
numerically seeded thermonuclear detonations that do not bear resemblance to physically
correct detonations. I then developed methods for ameliorating these problems, and
determined what metrics can be used for judging whether a given white dwarf merger
simulation is trustworthy. This involved the development of a number of algorithmic
improvements to the simulation software, which I describe. Finally, I performed high-
resolution simulations of typical cases of white dwarf mergers and head-on collisions to
demonstrate the impacts of these choices. The results of these simulations and the
corresponding implications for white dwarf mergers as astrophysical explosion
progenitors are discussed.

\newpage
\centerline{\bf{Table of Contents}}
\renewcommand*\contentsname{}
\tableofcontents

\newpage
\centerline{\bf{List of Figures/Tables/Illustrations}}
\vspace*{4\baselineskip}
Include these lists if applicable - if you have a list for each, each list must begin on a new page.

\newpage
\centerline{\bf{List of Abbreviations}}
\vspace*{4\baselineskip}
Include this list if applicable.

\newpage
\centerline{\bf{Acknowledgements}}
\vspace*{4\baselineskip}
When I graduated from high school, the event that mattered most to me was 
not walking across the stage at commencement, but rather the end-of-year 
debate team dinner and awards ceremony. The group of people that had been
my friends on the debate team made my experience much more memorable and 
gave me a connection to the school that I would not otherwise have had.
While I have lost touch with many of the people I knew at the time, I am 
still very close friends with my friends from debate. Andrew, Chris D.,
Chris V., Chris K., Lucas, and the rest of our group of friends have 
what we have in large part because of that activity, and that is what 
really made my educational activities complete.

In the same way, while I have been honored to have the opportunity to 
earn a doctorate from Stony Brook University, what really has made my 
time here special is the group of people I regularly interacted with 
on the astronomy floor. My advisor, Mike Zingale, is of course brilliant 
and someone whose talent level is extraordinary enough that I hope to 
one day reach even a fraction of that level. But I did not come to Stony 
Brook for that reason; there are smart and talented people everywhere. 
I came here and put in the effort to complete this dissertation because 
Mike inspires me to do better and constantly challenge myself;
and, because he shares many of my values about what makes good science and 
how good software development for scientific computing should be done;
but most importantly, because he has the necessary sense of humor to work 
with me. The other half of the dynamic duo that brought me here is Alan 
Calder, and he too played a key role in my success. Alan never let me 
doubt that I was capable of producing great scientific work, and would 
always drop whatever he was doing to chat with me about my concerns about 
my work or about my future. Many people have academic advisors, but some 
of them go above and beyond the call of duty to be true mentors and friends,
and Alan does that every day, not just for me but for all of the students 
he works with. Doug Swesty also played a key role in the development of the 
project that I worked on and in giving me many pointers on how to proceed 
in a novel way when I felt stuck. Even when he couldn't solve my problem, 
he always could provide some interesting insight into it, because there 
are few sharper and more thoroughly knowledgeable people out there.

Then were the students on the floor who made every day a delight, and who
were always willing to chat when I walked over to bother them (in some cases
a little \textit{too} willing, as anyone who was present for one of my many
silly arguments with Rahul can attest). Rahul and Melissa L. are good friends 
who helped keep the astronomy group anchored, and I am glad I got the 
opportunity to enjoy their presence. Adam and Don have a wealth of knowledge 
about subjects that nicely complements my own and working with them helped my 
own productivity enormously. Melissa H. also helped my productivity by doing 
a project that I never had time for -- and she is also a dear friend. Mat 
was also a great office mate, the few times we were actually in the office 
together. Thanks for sharing death row with me, buddy. And to everyone else
-- Lupe, Tianqi, Taeho, and everyone else who spent any time in the grad 
suite -- thanks for always being there to entertain my thoughts.

In my speech at the aforementioned debate dinner, I mentioned that my 
parents were the largest factor in my success, and that's just as true 
now as it was back then. I am truly fortunate to have parents who without 
hesitation will always talk to me, or let me crash at their house on 
last minute notice, or help me out financially when I need it. I don't 
see them or talk to them nearly as much as they would probably like, 
but they are always in my thoughts and I know that I am always in theirs.
And aside from the people already mentioned, Andrea played the most significant 
role in shaping the adult that I have become, and I am grateful to have known 
her for many years and to call her my friend.

Finally, I would like to acknowledge the support of Ann Almgren and Weiqun 
Zhang at Lawrence Berkeley National Laboratory's Center for Computational 
Science and Engineering. They were core developers of the software that 
I used for the simulations in this work, but they were also wonderful resources 
for sounding off new ideas and figuring out how to implement many of the 
algorithms I considered over the years.

\newpage
\pagenumbering{arabic}
\section{Introduction}
\label{sec:introduction}

Type Ia supernovae (SNe Ia) are among the most exciting
events to study in astrophysics. These bright, brief pulses of light
in the distant universe have led to a number of important discoveries
in recent years, including the discovery of the accelerated expansion
of the universe \citep{perlmutter1999,riess1998}. Their origin, though,
is shrouded in mystery. It has long been expected that these
events arise from the thermonuclear explosions of white dwarfs
\citep{hoyle-fowler:1960}, but the cause of these explosions is
uncertain. In particular, it is not clear what process causes the
temperatures in these white dwarfs (WDs) to become hot enough for explosive
burning of their constituent nuclei. The model favored initially by the
community was the single-degenerate (SD) model
\citep{whelan-iben:1973}. Accretion of material from a companion star
such as a red giant would cause the star to approach the Chandrasekhar
mass, and in doing so the temperature and density in the center would
become sufficient for thermonuclear fusion to proceed. In
recent years the focus has shifted to a number of alternative progenitor models. A
leading candidate for explaining at least some of these explosions is
the double-degenerate (DD) model, in which two white dwarfs merge and
the merged object reaches the conditions necessary for a thermonuclear
ignition \citep{ibentutukov:1984,webbink:1984}. Another is the double
detonation scenario, where accretion of material onto a
sub-Chandrasekhar mass white dwarf leads to a detonation inside the
accreted envelope, sending a compressional wave into the
core of the star that triggers a secondary detonation. A recent
review of the progenitor models can be found in
\citet{hillebrandt:2013}.

There are several observational reasons why double-degenerate systems
are a promising progenitor model for at least a substantial fraction
of normal SNe Ia. No conclusive evidence exists for a surviving
companion star of a SN Ia; this is naturally explained by the DD model
because both WDs are likely to be destroyed in the merger
process. Similarly, pre-explosion images of the SN Ia systems have
never clearly turned up a companion star, and in some cases a large
fraction of the parameter space for the nature of the companion star
is excluded. Additionally, not enough progenitor systems are seen for
the SD case to match the observed local SN Ia rate, whereas the number
of white dwarf binaries may be sufficient to account for this
rate. Finally, the DD model can naturally explain the fact that many
SNe Ia are observed to occur at very long delay times after the stars
were formed, since the progenitor systems only become active once both
stars have evolved off the main sequence. A thorough review of the
observational evidence about SNe Ia and further discussion of these
ideas can be found in \cite{maoz:2014}.

The first attempts to model the results of the merger process came in the
1980s. \cite{nomotoiben:1985} demonstrated that off-center carbon
ignition would occur in the more massive white dwarf as it accreted
mass near the Eddington rate from the less massive white dwarf
overflowing its Roche lobe. \cite{saionomoto:1985} tracked the
evolution of the flame and found that it propagated quiescently into
the center, converting the carbon-oxygen white dwarf into an
oxygen-neon-magnesium white dwarf. This would then be followed by
collapse into a neutron star---a result with significantly different
observational properties compared to a SN Ia. This scenario, termed
accretion-induced collapse, would be avoided only if the accretion
rate were well below the Eddington rate (see, e.g., \cite{fryer:1999}
for a discussion of the possible implications of the accretion-induced 
collapse scenario). \cite{tutukov-yungelson:1979}
observed that the collapse could be avoided if the mass loss from the secondary
was higher than the Eddington rate and thus the accreted material
formed an accretion disc, which might rain down on the primary more
slowly. The main finding was that double degenerate systems would not
obviously lead to Type Ia supernovae.

Three-dimensional simulations of merging double degenerate systems were 
first performed by \citet{benz:1990}, who used the smoothed particle
hydrodynamics (SPH) method to simulate the merger process. This was 
followed later by a number of authors 
\citep{rasio-shapiro:1995,segretain:1997,guerrero:2004,yoon:2007,loren-aguilar:2009,raskin:2012}.
The main finding of these early 3D SPH simulations was that if the 
lower-mass star (generally called the ``secondary'') was
close enough to the more massive star (the ``primary'') to begin mass
transfer on a dynamical time scale, the secondary completely disrupted
and formed a hot envelope around the primary, with a
centrifugally-supported accretion disk surrounding the core and
envelope. Carbon fusion might commence in the disk, but not at a 
high enough rate to generate a nuclear detonation. \cite{mochkovitch-livio:1990} 
and \cite{livio:2000}  also observed that turbulent viscosity in this disk 
would be sufficiently large for angular momentum to be removed from the 
disk at a rate high enough to generate the troublesome accretion 
timescales discussed by \cite{tutukov-yungelson:1979} and mentioned above. Based on this
evidence, the review of \cite{hillebrandtniemeyer2000} argued that the
model was only viable if the accretion-induced collapse problem could
be avoided. Later work by \cite{shen:2012} and \cite{schwab:2012} used
a more detailed treatment of the viscous transport in the outer
regions of the remnant and found that viscous dissipation in the centrifugally
supported envelope would substantially heat up the envelope on a  
viscous timescale, but their simulations still led to off-center carbon
burning. \cite{vankerkwijk:2010} argued that equal-mass mergers would
lead to the conditions necessary for carbon detonation in the center
of the merged object, but \cite{shen:2012} also questioned this for
reasons related to how viscous transport would convert rotational
motion into pressure support. \cite{zhu:2013} followed this with an
expanded parameter space study and argued that many of their
carbon-oxygen systems had the potential to detonate. The study of the
long-term evolution of the remnants is thus still an open subject of
research.

A recent shift in perspective on this problem started around 2010.
\cite{pakmor:2010} used the SPH method to study the merger of 
equal-mass ($0.9\ \msolar$) carbon-oxygen white dwarfs and found 
that a hotspot was generated near the surface of the primary 
white dwarf. They argued that this region had a temperature 
and density sufficient to trigger a thermonuclear
detonation. They inserted a detonation which propagated throughout 
the system. They found that the result would observationally 
appear as a subluminous Type Ia supernova. This was the first time 
a DD simulation successfully reproduced at least some characteristics of a SN
Ia. \cite{pakmor:2011} tried a few different mass combinations and
found empirically that this would hold as long as the secondary was at
least 80\% as massive as the primary. These events, where the merger
process resulted in the detonation of the system during the merger
coalescence---avoiding the much longer time-scale evolution---were
termed ``violent'' mergers.

Around the same time, however, \cite{guillochon:2010} and
\cite{dan:2011} pointed out that the previously mentioned simulations 
generally shared a significant drawback, which was that their initial conditions
were not carefully constructed. \cite{motl:2002}, \cite{dsouza:2006},
and \cite{motl:2007} (the first three-dimensional mesh-based
simulations of mass transfer in binary white dwarf systems) pioneered
the study of the long-term dynamical evolution of binary
white dwarf systems after constructing equilibrium initial
conditions. Earlier work placed the stars too close together 
and ignored the effects of tidal forces that change the shape of the 
secondary, leading to the merger
happening artificially too quickly \citep{fryer:2008}. When the initial conditions are
constructed in equilibrium, the system can be stable for tens of
orbital periods, substantially changing the character of the mass
transfer phase. One limitation of this series of studies is
that the authors used a polytropic equation of state and thus could
not consider nuclear reactions. \cite{guillochon:2010} and
\cite{dan:2011} improved on this using a realistic equation of state,
a nuclear reaction network, and a similar approach to the equilibrium
initial conditions, and found substantial agreement with the idea that
mass transfer occurs in a stable manner over tens of orbital
periods. They also found that, assuming the material accreted onto the
surface of the primary was primarily helium, explosive surface
detonations would occur as a result of accretion stream instabilities
during the mass transfer phase prior to the full merger. This could
trigger a double-detonation explosion and thus perhaps a SN Ia.

The latest violent merger developments have resulted in some possible areas of convergence.
\cite{pakmor:2012} performed a merger scenario
with a $1.1\ \msolar$ and $0.9\ \msolar$ setup, with better treatment
of the initial conditions, and indeed found that the merger process
happened over more than ten orbits. Nevertheless, they still determined
that a carbon-oxygen detonation would occur, in line with their
earlier results. \cite{moll:2014} and \cite{kashyap:2015} were also 
able to find a detonation in similarly massive systems. Notably,
the detonation occurred self-consistently and did not need to be  
intentionally triggered using an external source term.
\cite{dan:2012} and \cite{dan:2014} performed a large sweep 
of the parameter space for merger pairs and
found that pure carbon-oxygen systems would generally not lead to
detonations (and thus be violent mergers) except for the most massive
systems. They did find that for systems with WDs containing helium, many
would detonate and potentially lead to SNe Ia, either through the
aforementioned instabilities in the accretion stream, or during the
contact phase, similar to the violent carbon-oxygen WD
mergers. \cite{sato:2015} also examined the parameter space and
came to a similar conclusion for massive carbon-oxygen WD systems
(and also looked at the possibility of detonations after the
coalescence had completed), while \cite{tanikawa:2015} discussed
the plausibility of helium detonations in the massive binary case.
\cite{pakmor:2013} added a thin helium shell on their primary
white dwarf, and found that this robustly led to a detonation of the
white dwarf. For now there is preliminary support for the hypothesis
that systems with helium shells (or helium WDs), and very massive carbon-oxygen binaries,
could robustly lead to events resembling SNe Ia.

Given the considerable research into the double degenerate problem 
described above, why is another approach using a different simulation
code warranted? First and foremost, reproducibility of the results
across simulation codes and algorithms is important for gauging
confidence in this result. Most of the existing results that study 
the viability of double degenerate systems as progenitors for
Type Ia supernovae (that is, including a realistic 
equation of state and nuclear reactions) have
used the SPH method. SPH codes have a number of features which do aid
them in the study of these systems, such as conservation of
angular momentum to machine precision when there are no source terms
such as gravity (and conservation proportional to the level of
tolerance of error in the gravity solver when gravity is used).
A drawback relates to the fact that whether a prompt detonation
in a merger happens depends in detail on the nature of the
gas at the interface between the two stars, which is at much lower
density than the rest of the stellar material. The SPH codes for these
simulations generally all use
uniform mass particles, so their effective resolution is
\textit{lowest} at the stellar surface. In contrast, a code
with adaptive mesh refinement can zoom in on the regions where
hotspots will develop, while also maintaining high enough resolution
in the high-density regions to adequately capture the large-scale mass
transfer dynamics. There are also outstanding questions of
convergence in SPH (e.g.\ \citealt{zhu-SPH:2014}) and whether the method
correctly captures fluid instabilities. This is an important question
for white dwarf mergers because of the likely importance small-scale
instabilities will have on the evolution of the low-density gas at the
primary's surface. The pioneering work of \cite{agertz:2007} compared
grid and SPH codes and found some important differences. Most relevant
for this discussion is that the SPH codes could not adequately handle
mixing from the Kelvin-Helmholtz instability in the test they
propose. As pointed out by \cite{price:2008}, this is not a result of
SPH being inherently unable to model this instability, but instead it
is attributed to the fact that the standard SPH evolution equations do
not have a mechanism for capturing discontinuities in internal
energy. \citeauthor{price:2008} showed that the addition of an
artificial thermal conductivity can dramatically improve the ability
of the SPH codes to exhibit this instability. There have since been a
number of other papers discussing this issue, but to our knowledge
none of these improvements have yet been incorporated into an SPH
model of a WD merger. Another reason for caution is that other than the
most recent results of \cite{kashyap:2015}, no white dwarf merger simulation has self-consistently
resulted in a thermonuclear detonation. Reproducibility of the detonation 
through numerical simulation is critical for building 
confidence in this progenitor model.

This paper is the first in a series designed to address these
outstanding theoretical issues for white dwarf mergers. This work 
discusses the verification of our hydrodynamics code for simulating
these events. Later efforts will look at the initial conditions of the
system, the robustness with which a hotspot is found from which a
detonation could occur, and the importance of the initial white dwarf
models, which should be more sophisticated than simple carbon-oxygen
mixtures and in principle should use results from modern stellar
evolution calculations. \autoref{sec:methodology}
describes our code and why it can provide useful results compared to
other methodologies used for this problem. 
\autoref{sec:software} describes the method we use for setting up a
binary white dwarf simulation. \autoref{sec:verification} discusses a few
test problems that we use to verify that our code accurately
solves the equations of fluid dynamics. \autoref{sec:performance}
demonstrates that the software scales well for supercomputer
applications. In \autoref{sec:collisions} we discuss our results for
collisions of white dwarfs, and in \autoref{sec:mergers} we discuss
our results for mergers of white dwarsf. Finally, \autoref{sec:conclusion}
recaps what we have shown and highlights some of the future work we
plan to do.



\newpage
\section{Numerical Methodology}
\label{sec:methodology}

\subsection{Hydrodynamics}
\label{sec:hydrodynamics}

\subsection{Equation of State}
\label{sec:eos}

\subsection{Gravity}
\label{sec:gravity}

\subsection{Rotation}
\label{sec:rotation}

\subsection{Nuclear Burning}
\label{sec:burning}



\newpage
\section{Simulation Software}
\label{sec:software}



\newpage
\section{Verification Tests}
\label{sec:verification}



\newpage
\section{Parallel Performance}
\label{sec:performance}



\newpage
\section{White Dwarf Collisions}
\label{sec:collisions}

\subsection{Two-Dimensional Simulations}
\label{sec:collisions_2D}

\subsection{Three-Dimensional Simulations}
\label{sec:collisions_3D}



\newpage
\section{White Dwarf Mergers}
\label{sec:mergers}



\newpage
\section{Discussion and Conclusions}
\label{sec:conclusion}

\newpage
\bibliographystyle{aasjournal}
\bibliography{refs}

\end{document}
